\noindent \textbf{Title of paper:} ``Mastering Chess and Shogi by Self-Play with a General Reinforcement Learning Algorithm'' by Silver et.\ al.

\noindent\textbf{What is their primary result?} DeepMind has employed a variety
of distinct approaches in order to get machines to achieve superhuman levels of
performance in the games of chess, shōgi, and Go. The authors generalize the
distinct approaches to a single `AlphaZero algorithm.'

\noindent\textbf{Why is this important?} Previous approaches to the problem were
based on sophisticated search techniques handcrafted and refined by human
experts. A machine learning approach, AlphaGo, was recently applied to the game
of Go with much success. However, it is difficult to apply the AlphaGo's
neural network architecture to chess and shōgi because of the asymmetry of their
rules.

\noindent\textbf{What are their key ideas?} The authors generalize the AlphaGo
Zero algorithm to a more generic AlphaZero algorithm which uses deep neural
networks and a tabula rasa reinforcement learning algorithm. The neural network
in AlphaZero takes the board positions $s$ as input and outputs a vector of move
probabilities $\bfp$ with components $p_a=\Pr(a|s)$ for each action $a$, and a
scalar value $v$ estimating the expected outcome $z$ from position $s$. The
network optimizes these values entirely through self-play and uses these to
guide its search.

Instead of more traditional alpha-beta search with handcrafted enhancements,
AlphaZero uses a more general Monte Carlo tree search. Each search consists of a
series of simulated games and self-play that traverse a tree from root to leaf.
Each simulation proceeds by selecting in each state $s$ a move $a$ with low
visit count, high move probability and high value according to the current
neural network.
% \noindent\textbf{What are the limitations, either in performance or
%   applicability?} 

\noindent\textbf{What might be an interesting next step based on this work?} An
interesting next step would be to apply machine learning methods to more
sophisticated board games.

\noindent\textbf{What's the architecture?}

\noindent\textbf{How did they train and evaluate it?} They applied the AlphaZero
algorithm to chess, shogi, and Go. The training proceeded for $700\,000$ steps
in mini-batches of size $4\,096$ starting from randomly initialized parameters,
using $5\,000$ first-generation TPUs to generate self-play games and $64$
second-generation TPUs to train the neural networks.

They evaluated the trained instances of AlphaZero against Stockfish, Elmo, and
the previous version of AlphaGo Zero (trained for $3$ days) in chess, shōgi, and
Go, respectively. Alpha Zero defeated all opponents, losing zero games to
Stockfish, eight games to Elmo, and defeating the previous version of AlphaGo
Zero in all games.

% \noindent\textbf{Did they implement something?}

%%% Local Variables:
%%% TeX-master: "../MA598-DL-HW"
%%% End: