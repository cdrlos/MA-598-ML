\noindent \textbf{Title of paper:} ``'Explaining and Harnessing Adversarial
Examples'' by Goodfellow, Shlens, and Szegedy. 

\noindent\textbf{What is their primary result?} The authors argue that a neural
network's vulnerability to adversarial examples comes from its linear nature and
no the nonlinearity and overfitting. 

\noindent\textbf{Why is this important?} This work focuses on an aspect of the
problem which has not been previously addressed. Moreover, the point of view
yields simple and fast methods of generating adversarial examples and provide
examples for adversarial training. 

\noindent\textbf{What are their key ideas?}

\noindent\textbf{What are the limitations, either in performance or applicability?}

\noindent\textbf{What might be an interesting next step based on this work?}

\noindent\textbf{What's the architecture?}

\noindent\textbf{How did they train and evaluate it?}

\noindent\textbf{Did they implement something?}

% Adversarial examples are inputs to a neural network that result in an incorrect
% output from the network.

% Why this happens? Early attemps at explaining this problem focus on nonlinearity
% and overfitting and incufficient regularization.

% Hypothesize: Neural networks are too linear to resist linear adversarial
% perturbation.

% LSTMs, ReLUs , and maxout networks are all intentionally designated to behave in
% very linear ways, so that they are easier to optimize.

% Nonlinear models suchh as sigmoid networks are carefully.

% Introduction: Some previous work on adversarial examples.

% That is misclasisfy examlpes that are slightly different from correctly
% classified examples.
% A wid variety of models with different architecture trainde

% Open question.

% Why do adversanrial examples generalize?

% One adversarial example can attack multiple aversarial networks. 


%%% Local Variables:
%%% TeX-master: "../MA598-DL-HW"
%%% End: