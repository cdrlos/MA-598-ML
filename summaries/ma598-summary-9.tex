\noindent \textbf{Title of paper: } Empirical study of the topology and
geometry of deep networks.

\noindent\textbf{What is their primary result?}  The paper shows that
state-of-the-art deep networks learn connected classification regions, and that
the decision boundary in the vicinity of data points is flat along most
directions. Moreover, they show that the decision boundary is biased towards
negative curvatures; directions with significant curved decision boundaries are
shared between data points; and they demonstrate that there is a relationship
between the sensitivity of a classifier to perturbations and these shared
directions.
 
\noindent\textbf{Why is this important?} It is important to understand
the weaknesses of deep neural networks, specifically their instability under
perturbation. This paper addresses the geometric properties of such deep
networks to help shine a light on this problem.

\noindent\textbf{What are their key ideas?} Their key ideas are checking
the connectivity of the network. In particular, the CaffeNet architecture on
ImageNet classification. They find empirical evidence to conjecture that
classification regions are connected in $\R^N$.

Next they consider the curvature of the LeNet and NiN architectures trained on
the CIFAR-10 task, and show empirically that it is ``almost'' flat with a bias
towards negative curvature.

\noindent\textbf{What are the limitations, either in performance or applicability?}

\noindent\textbf{What might be an interesting next step based on this work?}

\noindent\textbf{What's the architecture?} The architectures used are
CaffeNet for the boundary regions and LeNet and NiN for the curvature analysis.

\noindent\textbf{How did they train and evaluate it?} These networks were
trained on ImageNet, and CIFAR-10, respectively.

\noindent\textbf{Did they implement something?} The authors implemented
two algorithms. Algorithm 1 finds a path between two datapoints, whereas
Algorithm 2 detects image perturbation.

%%% Local Variables:
%%% TeX-master: "../MA598-DL-HW"
%%% End: