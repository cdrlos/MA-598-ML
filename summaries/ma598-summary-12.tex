\noindent \textbf{\large Title of paper:} Audio Adversarial Examples: Targeted
Attacks on Speech Attacks.

\noindent\textbf{\large What is their primary result?} The primary result of the
paper is to demonstrate that the use of neural networks in audio recognition
tasks is vulnerable to adversarial attacks; that is, attacks where the main
objective of the attacker is to make the neural network classify an instance $x$
similar to a natural instance $y$ as any target $t$ chosen by the attacker.

\noindent\textbf{\large Why is this important?}

\noindent\textbf{\large What are their key ideas?}

\noindent\textbf{\large What are the limitations, either in performance or applicability?}

\noindent\textbf{\large What might be an interesting next step based on this work?}

\noindent\textbf{\large What's the architecture?}

\noindent\textbf{\large How did they train and evaluate it?}

\noindent\textbf{\large Did they implement something?}


% Automatic transcription is used all over the planet now using neural nets to
% figure out what audio clips are saying in words.

% Mozilla Deep Speech puts the shit into a Mel-Frequency Cepstrum.

% Given audio $x\in X$ and target $t$, authors conpute $\delta$ such that.

% The distortion induced by some perturbation $\delta$ is quatified by
% \[
%   dB_x(\delta)=dB(\delta)-dB(x),\quad \text{with}\quad
%   dB(x)=max_i [20\log_{10}(\abs{x_i})]
% \]

% One then solves the corresponding optimization problem with the two feasibility
% constraints.

% One then sloves the corresponding optimization problem with the two feasibility
% constraints.

% minimize $\delta$ over $dB_x(\delta)$ subject to $C(x+\delta)=t$,
% $x+\delta\in[-M,M]$. The authors test this on $100$ instances of Mozilla Common
% Voice dataset and target $10$ distinct incorrect transcriptions, chosen at
% random so that the transcription is incorrect, it is theoretically possible to
% reach ...

% The constraint $C(x+\delta)=t$ is nonlinear, which makes the original
% optimization problem hard. (How could

% Some notation: single input frame $\calX$ range of single frame $\calY$,
% $f\colon\calX^N\to[0,1]^{N\abs\calY}$. Probability frame $x_i\in\calX$ has label
% $j$.

%%% Local Variables:
%%% TeX-master: "../MA598-DL-HW"
%%% End: