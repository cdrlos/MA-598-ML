\noindent \textbf{Title of paper:} ``How SGD Selects the Global Minima in
Over-parameterized Learning: A Dynamical Stability Perspective'' by Wu, Ma, and
E.

\noindent\textbf{What is their primary result?} The authors propose a
dynamical-stability heuristic for analyzing the global minima selection of SGD
in the setting of over-parameterized learning (e.g., deep learning). Their
methods predict SGD convergence behavior which seems to correlate well with
previous empirical studies of the phenomenon.

\noindent\textbf{Why is this important?} Analyzing the selection behavior of SGD
contributes to our understanding of the algorithm and makes us more confident of
its applicability.

\noindent\textbf{What are their key ideas?} It has been previously observed that
SGD tends toward selecting \emph{flat minima}. In particular, work of
Jastrzębski et al.\ suggests that the ratio $\eta/B$ between the learning rate
and batch size is key to the selection of flat minima by SGD. Moreover, SGD
exhibits an escape behavior from global minima found by GD and toward minima of
smaller sharpness, i.e., smaller second derivative.

The authors introduce a dynamical stability perspective into the problem
(particularly, the notions of fixed point, linear stability, non-uniformity) and
prove that the global minimum $x^*$ (fixed point of a stochastic dynamics
problem) is linearly stable if a certain inequality involving the learning rate
$\eta$, batch size $B$, etc., is met.

% \noindent\textbf{What are the limitations, either in performance or applicability?}

% \noindent\textbf{What's the architecture?}

\noindent\textbf{How did they train and evaluate it?} The authors tested the
predictions of their method by carrying out two classification experiments: a
FNN with 898\,510 parameters on FashionMNIST and a VGG with 71\,410 parameters
on CIFAR10. Their experiments fit quite nicely their predicted behavior.

As predicted by their methods, GD tends towards the sharpest minima; in fact, in
both experiments the bound $\leq s/\eta$ (where $s$ is the sharpness) is very
nearly met. On the other hand, SGD tends to select flatter minima instead of
sharp minima with low non-uniformity. It is mentioned that as the non-uniformity
is reduced, the sharpness is reduced, so that there appears to be a strong
correlation between sharpness and non-uniformity.

\noindent\textbf{What might be an interesting next step based on this work?} In
their evaluation of their methods, the authors noticed a strong correlation
between sharpness and non-uniformity. Perhaps there is some theoretical reason
for this that yet to be explored.

% \noindent\textbf{Did they implement something?}

%%% Local Variables:
%%% TeX-master: "../MA598-DL-HW"
%%% End: